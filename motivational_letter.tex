\documentclass[12pt,a4paper, notitlepage]{report}
\usepackage[latin1]{inputenc}
\usepackage{amsmath}
\usepackage{amsfonts}
\usepackage{amssymb}
\usepackage{graphicx}
\usepackage{hyperref}
\usepackage{svg}

\title{Motivation Letter}
\author{Irvin Aloise}
\date{}


\begin{document}
\begin{flushright}
To whom it may concern, \\
\end{flushright}
\vspace{10px}

I am writing this letter regarding the possibility of joining Sapienza's robotics research team.

I hold a Bachelor Degree (B.Sc.) in Electronic Engineering (2010-2014) and I am going to finish my Master of Science (M.Sc.) in Artificial Intelligence and Robotics in October. 

My Bachelor Degree helped me to create a strong knowledge base on scientific fundamentals - e.g. physics, electronic principles and electromagnetic fields understanding. Moreover, this "first stage" was useful to make up my mind on how to tackle big problems and to solve them efficiently.

During my Master of Science, instead, I studied many aspects of robotics, but I was captured by autonomous and mobile robots in particular. For this reason, I chose to do many projects on this field, analyzing different aspects - e.g. \textit{Human-Robot Interaction}, \textit{Learning in Autonomous Systems}, \textit{Simultaneous Localization and Mapping}. Moreover, my master thesis focuses on SLAM back-end development and, thus, I was able to better analyze such topic, appreciating possible new results and future applications.

I think that this represents an opportunity for me of growing up in many aspects, both from the humanistic and the academic point of view. Thus, I would like to challenge myself starting this new research period and I believe that my motivation and background would help me to give a valuable contribution to the team. 

\vspace{5px}

I am really grateful for spending your time in reading this letter and I hope that you will take in consideration my application.

\vspace{20px}
Yours faithfully,
\vspace{5px}

\begin{flushright}
    \textit{Irvin Aloise}
\end{flushright}
\end{document}
